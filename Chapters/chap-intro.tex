\begin{flushright}
\textsl{To understand hydrogen is to understand all of physics.}\\
 \medskip{}
 --- unknown author \\
\bigskip{}
\textsl{You know, it would be sufficient to really understand the electron.}\\
 \medskip{}
 --- Albert Einstein
\par\end{flushright}

\bigskip{}
\begingroup
\let\clearpage\relax
\let\cleardoublepage\relax


\chapter{Introduction}

\label{ch:Introduction}
Understanding the interaction of ions with other particles is a
fundamental problem in atomic physics. The field of ion collisions
is very broad and includes interactions with photons, electrons,
atoms, molecules, clusters, or other ions. Our experimental work
focuses only on some very simple interacting systems like
$\Hminus+\Hyd$,
$\Dminus+\Hyd$, $\OHminus+h\nu$, or $\Htp+\elec$ and others. The
choice of considered systems follows from our interest in reactions
which are important in astrophysics. Also, it advantageous to study
the simple systems in order to obtain results directly comparable
against ab initio quantum calculations.

There are basically two main motivations for our work. Our first
aim is providing experimental values of rate coefficients and
cross sections which then serve as an input for kinetic models
in astrophysics or even in industrial plasma processing. The
other aim is a more fundamental quest for detailed understanding
of the underlying physics, where our results can serve as a benchmark
for various theoretical models.

Despite the fact, that a lot of experimental and theoretical effort
is dedicated to investigating much more complex systems, leading
mainly to organic chemistry and biomolecules, the are still many
unsolved problems even for the simple diatomic collisions. We
believe, that studying simple systems in the end provides
us with a better understanding in various fields from quantum mechanics
to atomic and molecular physics to plasma physics and to astrophysics.

\section{Ion-molecule reactions in astrophysics}
For decades, astronomers have been observing abundance of
various molecules and ions in the interstellar space. Currently
about 165 different molecular species have been detected in space, ranging
in size from simple diatomics to aromatic hydrocarbons to fullerenes 
\citep{muller2001,muller2005,cdms}.


With current advances in spectroscopical astronomical instruments such
as Planck \citep{planck2011}, Herschel \citep{herschel2010},
and especially ALMA \citep{semenov2008}, there is a rapidly growing
amount of data
regarding the molecular composition and conditions in the interstellar
medium. strong need for
g

Although it may seem---at first sight---that the evolution of large scale
celestial bodies is not greatly influenced by their chemical
composition and chemical processes occurring in the surrounding medium,
the astronomical observations and models suggest the exact opposite.
All the solid bodies in the Universe ultimately originate from
condensation of gaseous and occasionally dusty clouds of matter. An
important process in determining the behaviour of an astrophysical
cloud is the conversion between different forms of energy \eg\ between
macroscopic kinetic energy and thermal energy (friction) or between
kinetic energy of particles and radiation (cooling).

For example, the presence of ions in the protoplanetary disks is essential
to provide a coupling between the magnetic fields and matter, which leads
to magnetohydrodynamical instabilities \citep{balbus1991}. These in
consequence drive the
accretion and determine the sizes and orbits of the newly forming planets
(See \eg\ \cite{wardle2007,semenov2008} and references therein).

This work is motivated by another striking example of importance of chemistry
in astrophysics. To explain our motivation we need to look into the early phase
of universe evolution and into the formation of protogalaxies and
protostars in particular. Before the first stars (population III) were formed,
the elemental composition of the baryonic matter was determined by the \ac{BBN}
process.
The \ac{BBN}, which occurred during the first $1000\;$s after the Big Bang
\citep{walker1991}, produced
only small amounts of elements heavier than helium. The estimates of initial
composition of the baryonic matter are summarized in the table
\ref{tab:intro:bbn}.
\begin{table}
    \centering{
        \begin{tabular}{cc}
            nucleus & relative number to ${}^1$H\\
            \hline
            ${}^1$H       & $1$\\
           ${}^4$He       & $8.3\cdot10^{-2}$\\
            ${}^2$D       & $2.8\cdot10^{-5}$\\
           ${}^7$Li       & $\approx\cdot10^{-10}$\\
        \end{tabular}
    }
    \caption{Measured abundances of the elements after the Big Bang
    nucleosynthesis \citep{yao2006}. The primordial density of ${}^3$He has
    not been determined from observations, but the observations of more recent
    astronomical objects and models suggest a value of the order
    $10^{-5}$ \citep{balser1999}. Other elements are expected to be present at
    relative concentrations below $10^{-10}$ \citep{vangioni2000}.}
    \label{tab:intro:bbn}
\end{table}


Approximately $300\,000$~years after the Big Bang, the
energy of the cosmic background radiation decreased enough to allow for the
electron-proton recombination and formation of atoms. Only after the
recombination epoch, first chemical processes could be initiated. Due to the
elemental composition of the primordial gas, the chemistry was driven almost
purely by hydrogen interactions. 

The formation of first stars was triggered, according to the $\Lambda$CDM
cosmological model, by the presence of small spatial inhomogeneities in
the initial dark matter distribution (see \cite{abel2002,glover2011}).
These inhomogeneities underwent a
gravitational collapse and formed the so-called {\em dark matter halos}.
The dark matter halos of course attract the surrounding baryonic matter---the
primordial gas.
However, in general, a gravitationally compressed cloud of the primordial
gas heats up and the thermal pressure can prevent the cloud from further
collapse and formation of a protostar.
Only if there is a sufficiently
strong cooling process for a cloud of given mass,
the collapse can lead to a protostar formation.

Astrophysical models of star formation indicate, that a protostellar
cloud consisting of purely atomic species would need to be very massive
for the gravity to overcome the thermal pressure. This is because the
radiative
cooling by the electronic transitions in atomic species (\ie\ Lyman-$\alpha$
cooling in atomic hydrogen) is inefficient at temperatures below $10^4\;\text{K}$.
Only
a presence of molecular hydrogen in the primordial gas can provide sufficient
cooling by rovibrational transitions and explain
the formation of low-mass stars in the early universe \citep{palla1983}.

Since there are no significant amounts of elements heavier than helium in
the primordial gas, the molecular hydrogen can be formed in the gas phase
only, as opposed to the heterogeneous formation on dust surfaces in the more
recent molecular clouds. There are basically three main pathways to the
formation of the molecular hydrogen. At high redshifts, when the cosmic
background radiation energy was high enough to destroy most of \Hminus\ by
photodetachment, the main channel of \Htwo\ was based on the
charge transfer reaction \citep{glover2006}:
\begin{align}
    \hydrogen + \Hplus &\to \hydrogen_2^+ + h\nu\,,\\
    \hydrogen + \hydrogen_2^+ &\to \hydrogen_2 + \Hplus\,.
\end{align}
In the later phases, the \Hminus\ formed by the radiative attachment to \hydrogen\
was able to survive long enough to form molecular hydrogen by the associative
detachment reaction. Therefore, a sequence of reactions consisting of the radiative
attachment followed by the associative detachment
\begin{align}
    \electron + \hydrogen &\to \Hminus + h\nu\,,\\
    \Hminus + \hydrogen &\to \hydrogen_2 + \electron\,
    \label{eq:intro:AD}
\end{align}
represents the major pathway to the formation of the molecular hydrogen in a
broad range of conditions in the early
universe \citep{glover2006}. Only in the late phases of the protostellar cloud
collapse, when the number density of gas particles reaches
$n>10^8\;\text{cm}^{-3}$, the three-body processes take over and rapidly
convert the atomic hydrogen to \Htwo\ in reactions \citep{palla1983,turk2011}
\begin{align}
    \hydrogen + \hydrogen + \hydrogen &\to \Htwo + \hydrogen\,,\\
    \label{eq:intro:HHH}
    \hydrogen + \hydrogen + \Htwo &\to \hydrogen_2 + \electron\,.
\end{align}

Influence of the different processes leading to the formation of \Htwo\ 
has been studied using numerical models of protostar formation.
It has been shown for a range of initial conditions, that the outcome
of the models (\ie\ the
temperature and the density of the collapsing cloud at a certain time) strongly
depends on the rate of the \Htwo\ formation. In particular, \cite{glover2006} shows
that the uncertainty of the associative detachment \eqref{eq:intro:AD} reaction
coefficient leads to order of magnitude uncertainties in the models of
protostar formation from initially hot (TEMPERATURE) clouds. Furthermore,
also the ternary association reaction \eqref{eq:intro:HHH} was shown to have
great influence during the late phases of protostar formation \citep{turk2011}.
The uncertainty in the three body reaction coefficients  is several orders
of magnitude \citep{glover2008}, which seriously complicates our understanding
of the protostar formation process.
The situation with the associative detachment uncertainties has, however,
changed significantly since the publication by \cite{glover2006}. Details
of the associative detachment properties will be discussed in the next section.



\section{Ion-molecule reactions}
The importance of ion molecule reactions lies ....

\comment{tell something about high reactivity of ions in comparison
to neutrals, reaction barriers, langevin... Should read Cizek's
article again}

\section{Early universe chemistry}

Experimental techniques are an essential part
Early universe astrochemistry, formation of stars

Molecular clouds

Somehow separate this... Introduction should be $\approx1$ page, then some {\em astro} chapter.

\endgroup
