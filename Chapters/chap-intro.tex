\begin{flushright}
\textsl{To understand hydrogen is to understand all of physics.}\\
 \medskip{}
 --- unknown author \\
\bigskip{}
\textsl{You know, it would be sufficient to really understand the electron.}\\
 \medskip{}
 --- Albert Einstein
\par\end{flushright}

\bigskip{}
\begingroup
\let\clearpage\relax
\let\cleardoublepage\relax


\chapter{Introduction}

\label{ch:Introduction}
Understanding the interaction of ions with other particles is a
fundamental problem in atomic physics. The field of ion collisions
is very broad and includes interactions with photons, electrons,
atoms, molecules, clusters, or other ions. Our experimental work
focuses only on some very simple interacting systems like
$\Hminus+\Hyd$,
$\Dminus+\Hyd$, $\OHminus+h\nu$, or $\Htp+\elec$ and others. The
choice of considered systems follows from our interest in reactions
which are important in astrophysics. Also, it advantageous to study
the simple systems in order to obtain results directly comparable
against ab initio quantum calculations.

There are basically two main motivations for our work. Our first
aim is providing experimental values of rate coefficients and
cross sections which then serve as an input for kinetic models
in astrophysics or even in industrial plasma processing. The
other aim is a more fundamental quest for detailed understanding
of the underlying physics, where our results can serve as a benchmark
for various theoretical models.

Despite the fact, that a lot of experimental and theoretical effort
is dedicated to investigating much more complex systems, leading
mainly to organic chemistry and biomolecules, the are stil many
unsolved problems even for the simple diatomic collisions. We
believe, that studying simple systems in the end provides
us with a better understanding in various fields from quantum mechanics
to atomic and molecular physics to plasma physics and to astrophysics.

\section{Ion-molecule reactions}
The importance of ion molecule reactions lies ....

\comment{tell something about high reactivity of ions im comparison
to neutrals, reaction barriers, langevin... Should read Cizek's
article again}

\section{Early universe chemistry}

Experimental techniques are an essential part
Early universe astrochemistry, formation of stars

Molecular clouds

Somehow separate this... Introduction should be $\approx1$ page, then some {\em astro} chapter.

\endgroup
