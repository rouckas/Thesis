
\chapter{Modelling of charged particles}


In this chapter we will explain some basic principles of modelling
of the charged particle motion under the influence of other particles
and external electromagnetic fields. In the spirit of the previous chapter...

\comment{maybe put some sauce here}

\section{Analytic solutions -- drifts}
\comment{Cite Chen, Fitzpatrick}

Under influence of the electric field $\mb E$ and magnetic field $\mb B$,
the motion of a particle with a mass $m$ and a charge $q$ is governed by
the Newton equation with the Lorentz force
\begin{equation}
    m\frac{\de \mb v}{\de t} = q(\mb E + \mb v\times\mb B)\,.
\end{equation}
There are a few---exact or approximate---analytic solutions to this
equation in special configurations of electromagnetic fields. These
special cases are very important for understanding the behavior of the
charged particles in arbitrary electromagnetic fields. The solutions
relevant for this work will now be introduced for further reference.

We shall start our discussion with the simplest non-trivial case,
which is the case of constant homogeneous magnetic field
$\mb B = \mb B_0$ and zero electric field.
We can easily see that the Lorentz force is
perpendicular to the magnetic field, therefore the motion along
the field lines is unrestricted by the magnetic field. It can be further
observed, that the magnetic field can not do work on the particle,
because the Lorentz force is perpendicular to the velocity direction.
Therefore the particle energy remains constant. Obviously, a motion
with a constant force perpendicular to the direction of motion
yields a circular trajectory in the plane perpendicular to the
magnetic field, called {\em gyration}. The radius of gyration $r_g$, also
known as the Larmor radius, can be
determined by equating the magnitudes of the centripetal force and the
Lorentz force in the equation
\begin{equation}
    |q|v_\perp B = \frac{mv_\perp^2}{r_g}\,,
\end{equation}
where $v_\perp$ is the component of initial velocity perpendicular to $\mb B$
and $B$ is the magnitude of the magnetic field. The gyroradius is therefore
given by the equation
\begin{equation}
    r_g = \frac{mv_\perp}{|q|B}\,.
    \label{eq:part:rg}
\end{equation}
We immediately obtain the gyrofrequency, also known as cyclotron frequency, in
the form
\begin{equation}
    \omega_c = \frac{v_\perp}{r_g} = \frac{|q|B}{m}\,.
    \label{eq:part:omegac}
\end{equation}

To obtain an explicit formula for the trajectory, we can solve the
equation of motion directly. We orient the $z$ coordinate parallel
to the magnetic field $(\mb B=B\hat{\mb z})$ and the $x$ coordinate parallel
to the initial velocity of the particle $(\mb v_0 = v_\perp\hat{\mb x})$.
The origin of the coordinate system
will be the gyration center.
The equation of motion can be expressed in components as
\begin{equation}
    \dot v_x = \frac{qB}{m}v_y\,,\qquad
    \dot v_y = -\frac{qB}{m}v_x\,,\qquad
    \dot v_z = 0\,.
\end{equation}
By calculating the time derivative of the above equations,
identifying the gyrofrequency, and substituting
for $\dot v_x$ and $\dot v_y$ we obtain
\begin{align}
    \ddot v_x &= -\omega_c^2 v_x\,,\quad
              &\ddot v_y &= -\omega_c^2 v_y\,.
\end{align}
These harmonic oscillator equations have solutions which satisfy the initial
conditions and motion equations in the form
\begin{align}
    v_x &= v_\perp\cos(\omega_c t)\,,\quad
        &v_y &= -v_\perp\sign(q)\sin(\omega_c t)\,,
    \label{eq:part:gyrationv}
\end{align}
where $v_\perp$ is the
initial velocity component perpendicular to $\mb B$. Finally, integrating
the solutions for velocity yields the equations of trajectories, where we
can identify the previously defined gyroradius $r_g$
\begin{align}
    x &= r_g\sin(\omega_c t)\,,\quad
      &y &= r_g\sign(q)\cos(\omega_c t)\,.
    \label{eq:part:gyrationx}
\end{align}

If we superimpose constant uniform electric and magnetic fields $\mb E$
and $\mb B$ respectively, an analytic solution can still be obtained.
According to the previous discussion, the unrestricted motion parallel
to the magnetic field is described by a constant acceleration
$|q|E_\|/m$, where $E_\|$ is the component of the electric field
parallel to $\mb B$. Without loss of generality we now define $\mb E$ to
be perpendicular to $\mb B$. A solution to this problem can be obtained
without going into the details of the particle motion, just by considering
the transformation formulas for electromagnetic fields between two
inertial frames moving at relative velocity $\mb V$. The transformation
identities in the non-relativistic limit $V/c\ll1$ read
\begin{align}
    \mb E' &= \mb E + \mb V\times\mb B\,; & \mb B' = \mb B\,.
\end{align}
In a frame moving at velocity $\mb V = \mb E\times \mb B/B^2$, the
transformed field $\mb E'$ reduces to zero. Hence the motion in the
moving frame is the previously obtained gyration. In the non-moving frame,
the particle gyrates with a superimposed drifting motion---called
$\mb E\times\mb B$ drift---at a velocity
\begin{equation}
    \mb v_{E} = \frac{\mb E\times\mb B}{B^2}\,.
\end{equation}
Perhaps a more intuitive explanation to this drifting effect is obtained
by realizing that the gyroradius is proportional to the velocity.
Hence it is smaller in regions with higher
potential energy. During each gyration the particle travels a slightly
longer path on the low-potential side of its trajectory. This difference
accounts for the overall drifting motion.

This solution can be generalized to any force field $\mb F$ by substituting
$\mb F$ for $q\mb E$. The general drift velocity in the crossed fields is
\begin{equation}
    \mb v_{F} = \frac{\mb F\times\mb B}{qB^2}\,.
    \label{eq:part:crossdrift}
\end{equation}

Let us now move on to the case of an inhomogeneous magnetic field with a zero
electric field. In this case, analytic solutions do not exist and we shall
use the method of averaging.

Consider first a situation where the gradient
of the magnetic field is perpendicular to the field \ie\ $\nabla B\perp\mb B$.
Without loss of generality we orient the magnetic field along $z$ and
the gradient along $y$.
For the method of averaging to be applicable we assume that the magnetic
field changes
slowly on the scale of one gyroradius \ie
\begin{equation}
r_g\nabla B \ll B\,.
\end{equation}
The variation of $\mb B$ can now be treated as a perturbation. The lowest
order correction to the unperturbed trajectory is obtained by averaging
the Lorentz force $\mb F$ over one unperturbed orbit. It follows from the
symmetry
of the problem, that the only nonzero average component can be in the $z$
direction. Thanks to its slow variation, the magnetic field value can be
replaced by its Taylor expansion from the gyration center
\begin{equation}
    \mb B = \mb B_0 + (\mb r\cdot\nabla)\mb B + O(r^2)\,.
\end{equation}
The $y$ component of the Lorentz force
can be evaluated using the Taylor expansion
for $B$ in combination with the equations of the gyration orbit
\eqref{eq:part:gyrationx} and \eqref{eq:part:gyrationx}
\begin{equation}
    F_y = -qv_xB_z(y) = -qv_\perp\cos(\omega_c t)
    \left(B_0+r_g\sign(q)\cos(\omega_c t)\frac{\partial B}{\partial y}\right)\,.
\end{equation}
Averaging this equation over one gyration loop leads us to
\begin{equation}
    \left<F_y\right> = -qv_\perp r_g\sign(q)
    \frac{1}{2}\frac{\partial B}{\partial y}\,.
\end{equation}
Or more generally
\begin{equation}
    \left<\mb F\right> = -qv_\perp r_g\sign(q)
    \nabla B/2,.
\end{equation}
This average force again leads to a drifting motion which can be expressed
using equation \eqref{eq:part:crossdrift}
\begin{equation}
    \mb v_{\text{mag}} = v_\perp r_g \sign(q)\frac{1}{2}
    \frac{\mb B\times\nabla B}{B^2}
\end{equation}
This drift is called {\em magnetic} or {\em grad-B} drift.

At last we will discuss the effects of a gradient in $\mb B$ parallel to
the magnetic field
\ie $\nabla B\|\mb B$. This situation is especially important for our
experiment, which will be introduced in the next chapter.
Due to its zero divergence, the magnetic field cannot be parallel
to its gradient in a whole three dimensional
volume. We will assume that the condition is satisfied only on
the $z$ axis in cylindrical coordinates \ie\ $\mb B(r=0) \| \nabla B(r=0)$
and $\mb B(r=0, z)= B(r=0,z) \hat{\mb z}$. We further assume that the field is
rotationally symmetric around the $z$ axis.
The magnetic field must have a radial component,
which can be estimated in the vicinity of the $z$ axis using the
zero-divergence relation of the Maxwell equations in the cylindrical coordinates
\begin{equation}
    \frac{1}{r}\frac{\partial}{\partial r}(rB_r) + 
    \frac{\partial B_z}{\partial z} = 0\,,
\end{equation}
from which follows, assuming slowly varying field
\begin{equation}
    B_r\approx-\frac{r}{2}\left.\frac{\partial B_z}{\partial z}\right|_{z=0}\,.
    \label{eq:part:Br}
\end{equation}
The azimuthal component of the field is zero thanks to the rotational symmetry
and the absence of electric currents.
Similarly to the previous case, we define an unperturbed trajectory with
the gyration center on the $z$ axis. Due to the field symmetry, the only
net force acting on the test particle must be in the $z$ direction. The
$z$ component of the Lorentz force in the cylindrical coordinates is 
$F_z = -qv_\theta B_r$. Plugging into this expression $B_r$ from 
\eqref{eq:part:Br} yields
\begin{equation}
    F_z = \frac{1}{2}qv_\theta r\frac{\partial B_z}{\partial z}\,.
\end{equation}
It can be seen from the Lorentz force in our configuration,
that $qv_\theta = -|q|v_\perp$. Averaging of this constant expression
is trivial and inserting $r=r_g$ from expression \eqref{eq:part:rg}
leads to
\begin{equation}
    \left<F_z\right> = -\frac{1}{2}\frac{mv_\perp^2}{B}
    \frac{\partial B_z}{\partial z}\,.
\end{equation}
We now define the magnetic moment $\mu$ as
\begin{equation}
    \mu = \frac{1}{2}\frac{mv_\perp^2}{B}\,.
    \label{eq:part:mu}
\end{equation}
This force can be generalized in the form
\begin{equation}
    \mb F_\|  = -\mu(\hat{\mb B}\cdot\nabla)\mb B\,,
\end{equation}
where $\hat{\mb B}$ is a unit in the direction of $\mb B$. Defining a vector
magnetic dipole $\mb\mu = \mu\hat{\mb B}$ we see, that this is actually
the well known force exerted on a magnetic dipole
\begin{equation}
    \mb F_\|  = -({\mb\mu}\cdot\nabla)\mb B\,.
\end{equation}

It may seem that the magnetic field can do work through this
force---by accelerating particles along the field lines. We will show the
essence of this apparent work by looking at the behavior of the magnetic
moment during this acceleration.

We start from the motion equation for the gyration center. The equation
can be projected along $\mb B$ by scalar multiplication with $\hat{\mb B}$.
Like in the previous discussion we assume that the curvature of $\mb B$ is
negligible along the path. Therefore the projection can be exchanged with
differentiation leading to
\begin{equation}
    m\frac{\de{\hat{\mb B}\cdot\mb v}}{\de t}  =
    -\mu(\hat {\mb B}\cdot\nabla)B\,.
\end{equation}
We identify the parallel velocity $v_\| = \hat{\mb B}\cdot\mb v$ and multiply
the equation by $v_\|$ to obtain
\begin{equation}
    mv_\|\frac{\de{v_\|}}{\de t}  =
    -\mu (v_\|\hat {\mb B}\cdot\nabla)B\,.
\end{equation}
The left hand side can be rewritten as a derivative of energy. The right hand
side can interpreted as a convective derivative along the trajectory
\begin{equation}
    \frac{\de}{\de t}\left(\frac{1}{2}mv_\|^2\right)  =
    -\mu \frac{\de B}{\de t}\,.
\end{equation}
Using the law of energy conservation we can replace $v_\|$ by $v_\perp$
\begin{equation}
    \frac{\de}{\de t}\left(\frac{1}{2}mv_\perp^2\right)  =
    \mu \frac{\de B}{\de t}\,.
\end{equation}
Finally substituting the magnetic moment from the equation \eqref{eq:part:mu}
yields
\begin{equation}
    \frac{\de}{\de t}\left(\mu B\right)  =
    \mu \frac{\de B}{\de t}\,.
\end{equation}








\section{numerical integration}, HARHA

collisions - Coulomb, Langevin

Particle-in-cell???
