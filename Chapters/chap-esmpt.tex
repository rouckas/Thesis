
\chapter{The ES-MPT apparatus}

\label{ch:ESMPT}
\comment{here be something}

\section{Superimposed rf and magnetic field}
The problem of particle motion in a \ac{RF} field is by itself
relatively complicated. In the \ac{ES-MPT} we superimpose
the ion-trapping \ac{RF} field with an inhmogeneous magnetic
field for electron guiding, which further increases the complexity
of particle behavior. In order to assess the feasibility of
running the linear multipole \ac{RF} trap in a weak magnetic field
we investigate briefly the problem of ion motion in a superposition
of \ac{RF} and magnetic field. This problem can be viewed from
many different perspectives, each of which is suitable for
a certain range of physical conditions. Since we do not aim for a
general rigorous treatment, our discussion is based on study
of a specific simple geometry. We will see though, that most
of our observations could be generalized.

Possibly the simplest device featuring inhomogeneous RF electric
field is the cylindrical capacitor --- a hypothetical device
consisting of two infinitely long coaxial cylindrical electrodes.
The electric field between the electrodes follows from the Gauss' law
in the cylindrical coordinates
\begin{equation}
\mb E = C_E\frac{\hat {\mb r}}{r},
\end{equation}
where $\hat{\mb r}$ denotes a unit vector in the direction of the $r$
coordinate. The integration constant $C_E$ is determined by the
electrode dimensions and potentials.

For our test calculations, the following \ac{RF} field
parameters were used:
$C_E = 0.005\;\text{V}$; $\Omega=10\;\text{MHz}$.
Then assuming an \Hminus\ particle starting with zero
initial velocity at a distance of $r_0=50\;\mu\text{m}$ from the
axis, the adiabaticity parameter given by equation
\eqref{eq:trap:adiab} takes the value $\eta\approx 0.15$, which
is safely below the critical threshold $0.3$.
