
\chapter{Modelling of electromagnetic fields}

\label{ch:Modelling}In principle almost all processes occuring in
the fields of molecular and atomic physics can be reduced to electromagnetic
interaction of elementary particles. However, for the purpose of this
chapter we limit our discussion only to large scale electrostatic
and magnetostatic fields. These are determined in the experiment by
applied potentials to the electrodes, currents in the electromagnets,
and spatial distribution of charged particles in case of plasmatic
experiments. Proper understanding of large scale electromagnetic fields
is essential for any experiment involving charged particles. In this
section, we explain our use of numerical models to obtain magnetic
and electric fields as a solution to Maxwell's equations

\begin{eqnarray}
\nabla\cdot\mathbold{D} & = & \rho\\
\nabla\cdot\mathbold{B} & = & 0\\
\nabla\times\mathbold{E} & = & \frac{\partial\mathbold{B}}{\partial t}\\
\nabla\times\mathbold{H} & = & \mathbold{J}+\frac{\partial\mathbold{D}}{\partial t}\text{{,}}
\label{eq:sim:maxwell}
\end{eqnarray}
in combination with material relations
\begin{equation}
\mathbold{D}=\epsilon\mathbold{E},\;\mathbold{H}=\mathbold{B}/\mu,\;\mathbold{J}=\sigma\mathbold{E}\text{.}
\label{eq:sim:materials}
\end{equation}
In the case of static fields, the time derivatives can be neglected.
In our experimental setups, the electrostatic field is completely
determined by the electrode potentials and space charge densities
inside the vacuum vessel. Influence of induced currents on the magnetic
field is negligible in comparison with the external magnetic field
generated by electromagnets. Hence, the electrostatic and magnetostatic
fields are not coupled in any way. The separate treatment of electric
and magnetic fields will be presented in the next sections. 

Please note, that we do not discuss complete mathematical foundations of the involved numerical methods. Only the principles and the 
information necessary to implement these methods with help of
freely available libraries are. For detailed discussion refer to any 
textbook \cite{fenicsbook}.


\section{Electrostatic fields}

The governing equation for electrostatic field is the the Poisson
equation. Since we are interested in the field in vacuum, we can substitute
for $\mathbold{D}$ from the material equation (\ref{eq:sim:materials}). Then using the relation for the scalar potential $\mb E = -\nabla \phi$ we obtain
\begin{equation}
\Delta\phi=-\frac{\rho}{\epsilon_{0}}\text{.}
\end{equation}
Now in order to solve for the potential $\phi$ in certain domain $\Omega$, we need to specify the charge density $\rho$ in this domain and the boundary conditions (Dirichlet, Neumann, or mixed) on the domain boundary $\partial\Omega$.



\section{Magnetostatic fields}
